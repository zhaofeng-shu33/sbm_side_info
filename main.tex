\documentclass[conference]{IEEEtran}
\usepackage{bm}
\usepackage{bbm}
\usepackage{amsmath}
\usepackage{amsthm}
\usepackage{algorithm,algorithmic}
\usepackage{url}
\usepackage{authblk}
\usepackage{amssymb}
\usepackage{mathtools}
\DeclarePairedDelimiter\norm{\lVert}{\rVert}
\DeclareMathOperator{\Var}{Var}
\newtheorem{definition}{Definition}
\newtheorem{theorem}{Theorem}
\newtheorem{lemma}{Lemma}
\newtheorem{remark}{Remark}
\newtheorem{corollary}{Corollary}
\DeclareMathOperator{\SSBM}{SSBM}
\DeclareMathOperator{\SDP}{SDP}
\DeclareMathOperator{\Tr}{Tr}
\DeclareMathOperator{\E}{\mathbb{E}}
\DeclareMathOperator{\diag}{diag}
\DeclareMathOperator{\dist}{dist}
\DeclareMathOperator{\Bern}{Bern}
\DeclareMathOperator{\Binom}{Binom}
\DeclareMathOperator{\KL}{KL}

\newcommand{\A}{\frac{a \log(n)}{n}}
\newcommand{\B}{\frac{b \log(n)}{n}}
\title{Exact Recovery in the Balanced Stochastic Block Model with Side Information}
\author[1]{\textbf{Jin Sima}}
\author[2]{\textbf{Feng Zhao}}
\author[3]{\textbf{Shao-Lun Huang}}
\affil[1]{\normalsize{Department of Electrical Engineering, California Institute of Technology, Pasadena 91125, CA, USA}}
\affil[2]{\normalsize{Department of Electronic Engineering,
                    Tsinghua University, 
                    Beijing, China 100084}}
\affil[3]{\normalsize{DSIT Research Center,
                    Tsinghua-Berkeley Shenzhen Institute,
                    Shenzhen, China 518055}}
\begin{document}
\maketitle
\begin{abstract}
	The role that side information plays in improving the exact recovery threshold in the stochastic block model (SBM) has been studied in many aspects. This paper studies 
	exact recovery in balanced binary symmetric stochastic block models with side information, given in the form of i.i.d. samples at each node. Compare to existing works, the balanced constraint in the SBM gives us two benefits. First, we derive a sharp exact recovery threshold that is analytical. Second, we present an efficient semi-definite programming (SDP) algorithm that achieves the optimal exact recovery threshold. Our SDP algorithm is a non-trivial generalization of the SDP algorithm for SBM without side information in the sense that our proof involves more detailed arguments.	
% The role that side information plays in improving the exact recovery threshold in the stochastic block model (SBM) has been studied in many aspects. This paper studies 
% 	exact recovery in balanced binary symmetric stochastic block models with side information, given in the form of i.i.d. samples at each node.
% 	The balanced constraint in the SBM allows us to derive a sharp exact recovery threshold that is analytical. We also present an efficient semi-definite programming (SDP) algorithm that achieves the optimal exact recovery threshold. Our SDP algorithm is a non-trivial generalization of the SDP algorithm for SBM without side information in the sense that our proof involves more detailed arguments. Moreover, the SDP algorithm also applies to the binary SBM where the label of each node has a uniform distribution.
\end{abstract}
\section{Introduction}
In network analysis, community detection assigns discrete labels to each node of the graph based on the observation of graph edges.
In addition to edge information, extra node features are often available in real-world applications in the form of graph signal \cite{dong2020graph},
noisy labels \cite{mossel2016local}, or
feature vectors \cite{zhang2016community}. Combining the edge and node information, it is expected that better
accuracy can be achieved for community detection problems. Within this context, a central problem 
is to investigate the gain that extra information brings to the detection problem, compared to the case when only edge observation is available.

	% first paragraph: short intro to SBM and Ising model
To get theoretical insight into such a problem, it is often assumed that the graph is generated from a simple probabilistic model called Stochastic Block Model (SBM), in which the probability of edge existence is higher within the community than between different communities \cite{holland1983stochastic}. For the solely presence of SBM, the condition on exact recovery of community labels has been studied extensively and the phase transition property has been established \cite{abbe2015community, mossel2016}. For a special case of two community model,
the recovery condition is summarized as $\sqrt{a} - \sqrt{b} > \sqrt{2}$ when $a,b$ are parameters of SBM.

With the presence of extra node information, the condition of exact recovery is improved
and generalized \cite{saad2018community, abbe17sideinfo}. However, previous study does not exactly quantify the contribution of side information and graph information. In contrast, this paper will fill the gap by considering a model of two-community SBM with extra node feature vectors. Our result generalizes
the exact recovery threshold to a condition $\gamma D_{1/2}(p_0 || p_1) + (\sqrt{a} - \sqrt{b})^2 > 2$
where the contribution of side information is coded in Renyi divergence.

To achieve the exact recovery condition of SBM, semi-definite programming (SDP) is often utilized \cite{Hajek16}.
SDP relaxation can also be used for SBM with side information \cite{esmaeili2019exact}, and in this paper we show that a sub-optimal exact recovery condition
of our model is achievable by such method.

This paper is organized as follows. In Section \ref{s:rw}, we review the previous works which are closely related with ours.
In Section \ref{s:model}, we introduce the model and present our main results.
Then the article concludes in Section \ref{s:conclusion} and
detailed proofs are provided in Section \ref{s:proof}.

The following notations are used throughout this paper: 
the random undirected graph $G$ is written as $G(V,E)$ with vertex set $V$ and edge set $E$;
$V=\{1,\dots, n\} =: [n]$;
$\mathcal{X}$ is the alphabet
of the random variable $X$; $m$ is the number of samples generated at each node;
$\Bern(p)$ and $\Binom(n,p)$ represent Bernoulli
and Binomial distribution respectively; $f(n)=\omega(g(n))$(or $=o(g(n))$) means that $\lim_{n\to \infty} f(n) / g(n) = \infty $(or $=0$);
$\mathbbm{1}[A]$ is the indicator function for the event $A$; $W^n$ is the n-ary Cartesian power of the set $W$;
The Hamming distance of 
two $n$-dimensional vectors is written as $\dist(x,y):=\sum_{i=1}^n \mathbbm{1}[x_n\neq y_n]$ for $x,y\in \{\pm 1 \}^n$.

\section{Related Works}\label{s:rw}
This work extends the model of two-community SBM considered in \cite{abbe2015community}.
Specifically, we assume the extra feature vectors of each node are independent samples, whose distribution depends on the label of the node.
This model has been studied in Section V-B of \cite{saad2018community}. However,
\cite{saad2018community} only got a weak conclusion, which says that the sample complexity of feature vectors
$m$ is required to be of order $O(\log n)$ for side information to take effects. In this paper, we obtain
a closed-form condition for exact recovery when $m=\gamma \log n$ for a positive constant $\gamma$.

A general case of side information is studied
in \cite{abbe17sideinfo}. We emphasize that the model setting in Theorem 4 of \cite{abbe17sideinfo}
assumes that the node labels are independently generated  from $\Bern(\frac{1}{2})$ while the model
in this paper requires uniform distribution over the space $\sum_{i=1}^n Y_i = 0$ where $Y_i \in \{\pm 1 \}$ is the label of the $i$-th node.
Although these two settings are equivalent in
SBM model when $n$ is large, we observe that it differs when side information is available. Our assumption is easier to analyze due to some
symmetric property of node observations.

Rényi divergence has been used in SBM in \cite{zhang2016} to characterize the weak recovery error bound. Both the dense and sparse graph are considered.
Within this paper, we use Rényi divergence to characterize
the contribution of side information to exact recovery error bound in these two cases.
\section{Preliminaries}\label{s:model}
A balanced binary symmetric stochastic block model (BSBM) is defined by a random graph $Z=\{Z_{i,j}\}_{1\le i,j\le n}$ with $n$ nodes $\{1,\ldots,n\}$ and edges $\{Z_{i,j}\}_{1\le i,j\le n}$, where $Z_{i,j}=1$ if nodes $i$ and $j$ are connected with an edge and $Z_{i,j}=0$ otherwise. 
Each node $i$, $i\in\{1,\ldots,n\}$ is associated with a label $Y_i\in \{\pm 1\}$ such that the labels $Y=(Y_1,\ldots,Y_n)$ is uniformly distributed over the space $\{Y:\sum^n_{i=1}Y_i=0\}$. The edges $\{Z_{i,j}\}_{1\le i,j\le n}$ are independently distributed Bernouli random variables, where $Z_{i,j}=1$ with probability $p=a\frac{\log n}{n}$ for nodes $i,j$ with the same labels, i.e., $Y_i=Y_j$, and $Z_{i,j}=1$ with probability $q=b\frac{\log n}{n}$ if $Y_i\ne Y_j$. In this paper, it is assumed that $a>b$. Since the probability distribution of the graph is determined by $n$, $p$ and $q$,
the BSBM model is denoted as $BSBM(n,p,q)$. 



A BSBM with side information is a generalization of  $BSBM(n,p,q)$. In addition to the graph $Z$ and the labels $Y$, each node $i$ has $m=\gamma \log n$ data samples $X^i_{j}$, $i\in \{1,\ldots,n\}$, $j\in \{1,\ldots,m\}$, that are drawn identically and independently from distribution $P_0$ if $Y_i=1$ and from distribution $P_1$ if $Y_i=-1$. Note that the data samples $X^i_{j}$, $j\in \{1,\ldots,m\}$ are independent from $\{Z_{i,j}\}_{1\le i,j\le n}$ given the label $Y_i$ for any $i\in\{1,\ldots,n\}$. Hence, the joint probability distribution of $(\{Z_{i,j}\}_{1\le i,j\le n},\{X^i_{j}\}_{1\le i\le n,1\le j\le m})$ conditioned on $Y$ is given by
\begin{align}\label{eq:lh}
    &P(x=\{x^i_{j}\}_{1\le i\le n,1\le j\le m},z=\{z_{i,j}\}_{1\le i,j\le n}| (y_1,\ldots,y_n)) \nonumber\\
    =& \prod_{1\le i,j\le n}P(z_{i,j}|y_i,y_j)\prod_{i=1}^n \prod_{j=1}^m P(x^i_j|y_i), 
\end{align}
where 
	\begin{equation*}
	P  (z_{i,j}=1|y_i,y_j) = \begin{cases}
	p & \text{if } y_i=y_j \\
	q & \text{if } y_i\ne y_j
	\end{cases},
	\end{equation*}
and
	\begin{equation*}
	 P(x^i_j|y_i) = \begin{cases}
	P_0(x^i_j) & Y_i = 1 \\
	P_1(x^i_j) & Y_i = -1
	\end{cases}
	\end{equation*}
The conditional probability distribution $P(\{x^i_{j}\}_{1\le i\le n,1\le j\le m},\{z_{i,j}\}_{1\le i,j\le n}| y_1,\ldots,y_n)$ is determined by parameters $n$, $p$, $q$, $P_0$, and $P_1$. Hence, 
the BSBM with side information is denoted as $SSBM(n,p,q,P_0,P_1)$.
In BSBM with side information $SSBM(n,p,q,P_0,P_1)$, the goal is to recover the unknown labels $Y$, given the graph $Z$ and the data samples $X$. In this paper, we consider exact recovery of $Y$, which is defined as follows.
\begin{definition}[Exact Recovery for $SSBM(n,p,q,P_0,P_1)$]
		Let
		$(Z=\{Z_{i,j}\}_{1\le i,j\le n},Y,X=\{X^i_{j}\}_{1\le i\le n,1\le j\le m})$ be a graph $Z$, node labels $Y$, and node data samples $X$ be drawn from the distribution defined by $SSBM(n,p,q,P_0,P_1)$.
		Exact recovery is solvable if there exists an algorithm that takes $(Z,X)$ as inputs and outputs $\hat{Y}=\hat{Y}(Z,X)$ such that the error probability $P_e:=P(\hat{Y} \neq Y)$ goes to $0$ as $n$ increases.
\end{definition}
A key measure used throughout this paper is the Rényi divergence, defined by
\begin{equation}
D_{1/2}(P_0 || P_1) \triangleq -2\log(\sum_{x \in \mathcal{X}} \sqrt{P_0(x)P_1(x)} ),
\end{equation}
where $P_0$ and $P_1$ are the distributions defined in the balanced stocastic block model with side information $SSBM(n,p,q,P_0,P_1)$.
The Rényi divergence characterizes the error exponent of a SSBM when the graph observation $Z$ is not  available and exact recovery of the labels $Y$ is done with only side information $X$. The following property of the Rényi divergence will be used in proving our main results. It can be proved by using the Lagrange multiplier.
\begin{lemma}\label{lem:p0p12}
	Let $p_0, p_1$ be probability distribution functions defined over $\mathcal{X}$. The minimizer
	of $D(X||p_0) + D(X||p_1)$ for any discrete random variable $X$ is
	\begin{equation}\label{eq:p012}
	P(X=x)=\frac{\sqrt{p_0(x)p_1(x)}}{ \sum_{x\in \mathcal{X}} \sqrt{p_0(x) p_1(x)}}
	\end{equation}
	and the minimal value is
	$-2\log \sum_{x\in \mathcal{X}} \sqrt{p_0(x) p_1(x)}$.
\end{lemma}
\section{Sharp Threshold for Balanced SSBM}
In this section we present an analytic sharp threshold for exact recovery in the balanced SSBM  We first provide the threshold, that comes from
 the phase transition in the exact recovery in SSBM with sparse graph structure. 
\begin{theorem}\label{thm:Pe}
For a balanced $SSBM(n,p=a\frac{\log n}{n},q=b\frac{\log n}{n},P_0,P_1)$, exact recovery is solvable if
\begin{equation}\label{eq:positive_condition}
\gamma D_{1/2}(P_0||P_1) + (\sqrt{a} - \sqrt{b})^2 > 2,
\end{equation}
and is not solvable if $\gamma D_{1/2}(p_0||p_1) + (\sqrt{a} - \sqrt{b})^2 < 2$, in which case, the error probability $P_e$ goes to $1$.
\end{theorem}

engineering insight

Theorem \ref{thm:Pe} shows an improvement of the exact recovery threshold when side information presents, over a BSBM where no side information is available. Moreover, the threshold implies a decoupling effect of graph observation and side information, in the sense that the left part of Eq. \eqref{eq:positive_condition} is exactly the sum of the terms that appear in the exact recovery threshold in SSBM when there is graph observation $Z=\{Z_{i,j}\}_{1\le i,j\le n}$ only and when there is side information $X=\{X^i_j\}_{1\le i\le n, 1\le j\le m}$ only, respectively. This decoupling is in contrast to the case in \cite{abbe17sideinfo}, where side information $X$ and graph observation $Z$ are coupled and a joint optimization in both $Z$ and $X$ is needed.  
\subsection{Proof of Theorem \ref{thm:Pe}}
%\begin{proof}[Proof of Theorem \ref{thm:Pe}]
The proof of Theorem \ref{thm:Pe} follows similar arguments to those in \cite{abbe2015exact}. We first show the first part of Theorem \ref{thm:Pe}, i.e., the achievability of exact recovery when \eqref{eq:positive_condition} holds. For any label $Y=(Y_1,\ldots,Y_n)$ satisfying $\sum^n_{i=1}Y_i=0$, let $A(Y)=\{i:Y_i=1\}$ and $B(Y)=\{i:Y_i=-1\}$ be the sets of nodes with labels $1$ and $-1$, respectively.  
Let $Y^*=(y^*_1,\ldots,y^*_n)$ be the true labels of the nodes. We prove that with high probability, $Y^*$ is the output of an MAP (maximum a posteriori probability) estimator, which is equivalent to an ML (maximum likelihood) estimator since $Y$ is uniformly distributed over the space $\{Y:\sum^n_{i=1}Y_i=0\}$.
Let $F_k$ be the event that there exists a labeling $Y'$ satisfying $|A(Y')\backslash A(Y)|=|B(Y')\backslash B(Y)|=k$. Then, by the union bound, it suffices to show that the probability $P(F_k)$ of the event $F_k$ is upper bounded by $o(\frac{1}{n})$ for anly $k\in\{1,\ldots,n\}$.

Let $P_k$ be the maximum probability that 
the likelihood $P(X,X| (Y^*_1,\ldots,Y^*_n)$ is smaller than the likelihood $P(X,Z| (Y'_1,\ldots,Y'_n))$, 
over all labeling $Y'=(Y'_1,\ldots,Y'_n)$ satisfying $|A(Y')\backslash A(Y)|=|B(Y')\backslash B(Y)|=k$, where the likelihood $P(X,Z| (Y_1,\ldots,Y_n)$ is defined in \eqref{eq:lh}. Then by the union bound, we have that 
\begin{equation}\label{eq:FAk}
P(F_k) \leq \binom{n/2}{k}^2 P_k
\end{equation}
It remains to derive an upper bouond on $P_k$. 
According to \eqref{eq:lh}, $P_k$ is the probability that
\begin{align}\label{eq:ein}
&\sum^m_{j=1}(\sum_{i\in A(Y^*)\backslash A(Y')} \log \frac{P_1(X^i_{j})}{P_0(X^i_{j})}
+\sum_{i\in A(Y')\backslash A(Y^*)} \log \frac{P_0(X^i_{j})}{P_1(X^i_{j})})\nonumber\\
\ge & \log \frac{p(1-q)}{q(1-p)} (\sum_{i\in A(Y^*)\backslash A(Y'),j\in A(Y^*)\cap A(Y')}z_{i,j}\nonumber\\
&+\sum_{i\in A(Y')\backslash A(Y^*),j\in B(Y^*)\cap B(Y')}Z_{i,j}\nonumber\\
&-\sum_{i\in A(Y^*)\backslash A(Y'),j\in B(Y^*)\cap B(Y')}Z_{i,j}\nonumber\\
&-\sum_{i\in A(Y')\backslash A(Y^*),j\in A(Y^*)\cap A(Y')}Z_{i,j})
\end{align}
Note that the right hand side of \eqref{eq:lh} can be written as $\log \frac{p(1-q)}{q(1-p)}(Z_1-Z_2)$, where $Z_1\sim Binom(k(n-2k),\frac{a\log n}{n})$ and $Z_2\sim Binom(k(n-2k),\frac{b\log n}{n})$ 
follow binomial distributions, where a binomial distribution $Binom(s,r)$ is the probability distribution of the sum of $s$ independent Bernouli random variables, each equals $1$ with probability $r$.
The left hand side can be expressed as
\begin{align*}
\epsilon\triangleq&\log^{-1}( a /b)\cdot [\sum_{i\in A(Y^*)\backslash A(Y')}(D(P_{\widetilde{X}^i} || P_0) - D(p_{\widetilde{X}^i} || P_1)) \\
&+ \sum_{i\in A(Y')\backslash A(Y^*)}(D(p_{\widetilde{X}^i} || P_1) - D(p_{\widetilde{X}^i} || P_0))],
\end{align*}
where $P_{\widetilde{X}^i}$ is the empirical distribution of samples $\{x^i_j\}^m_{j=1}$ at node $i$. 
\begin{lemma}\label{lem:zxt}
	Suppose $m > n, Z \sim \Binom(m, \frac{b\log n}{n}), X\sim \Binom(m, \frac{a\log n}{n})$.
	For $ t \geq \frac{m}{n}(b - a)$, we have
	\begin{equation}\label{eq:estimation}
	P(Z - X \geq t \log n) \leq \exp(-\frac{m}{n}\log n \cdot ( g(a, b, \frac{n}{m}t) + O(\frac{\log n}{n})))
	\end{equation}
	where $g(a,b,\epsilon)$ is defined as
	\begin{equation}\label{eq:gab}
	g(a,b,\epsilon) = a + b - \sqrt{\epsilon^2 + 4ab} + \epsilon \log \frac{\epsilon + \sqrt{\epsilon^2 + 4ab}}{2b}
	\end{equation}
\end{lemma}



\begin{proof}[Proof of Theorem \ref{thm:Pe}]
We start from \eqref{eq:ein}.
Since $m=\gamma \log n$, $\sum_{i=1}^{km} \log \frac{p_1(x_{1i})}{p_0(x_{1i})} \sim
-km D(p_0 || p_1)$, which is of order $O(\log n)$.
Let $\epsilon$ be defined as follows:


Then \eqref{eq:ein} is equivalent to
\begin{equation}\label{eq:zeps}
\sum_{i=1}^{k(n-2k)}(z'_{i} - z_{i}) \geq \epsilon \log n
\end{equation}
Using the type theory (11.1 of \cite{cover1999elements}), we can estimate the probability of the event $A_k$ given by \eqref{eq:zeps}.
\begin{align}
P(A_k) & =  \sum_{P^{(1)},P^{(2)}\in \mathcal{P}_{km}} Q_0^{km}(T(P^{(1)}))Q_1^{km}(T(P^{(2)}))\notag \\
& \cdot P(\sum_{i=1}^{k(n-2k)} (z'_{i} - z_{i} \geq \epsilon(P^{(1)}, P^{(2)})  \log n ))\label{eq:decomp} 
\end{align}
Then using Theorem 11.1.4 of \cite{cover1999elements} and Lemma \ref{lem:zxt} 
\begin{align*}
&P(A_k)  \leq \sum_{P \in P_n} \exp(-km (D(p_{\widetilde{X}_1} || p_0) + D(p_{\widetilde{X}_2} || p_1))) \\
& \cdot \exp(-\log n \frac{k(n-2k)}{n}\cdot (g(a, b, \frac{n}{k(n-2k)}\epsilon) + o(1))) \\
&\leq |\mathcal{P}_{km}|^2 \exp(-k\log n \cdot (\theta^*_k + o(1))) 
\end{align*}
where
\begin{align}
\theta^*_k &= \min_{\widetilde{X}_1,\widetilde{X}_2} \gamma (D(p_{\widetilde{X}_1}|| p_0) + D(p_{\widetilde{X}_2} || p_1)) \notag \\
&\quad + (1-\frac{2k}{n}) g(a,b, \frac{n}{k(n-2k)}\epsilon) \label{eq:theta_star}
\end{align}
We can verify $\frac{\partial^2 g(a,b,\epsilon)}{\partial \epsilon^2} =\frac{1}{\sqrt{4ab+\epsilon^2}}> 0$,
a lower bound of $\theta^*$ is obtained by using linear expansion of $g(a,b, \epsilon)$:
\begin{equation}\label{eq:g_linear}
g(a,b,\epsilon) \geq  (\sqrt{a} - \sqrt{b})^2 + \frac{\epsilon}{2}\log \frac{a}{b} 
\end{equation}
By Lemma \ref{lem:p0p12}, we can get
\begin{align}
\theta^*_k &\geq (1-\frac{2k}{n})(\sqrt{a} - \sqrt{b})^2
+ \frac{ \gamma}{2}\min (D(p_{\widetilde{X}_1}||p_0) + D(p_{\widetilde{X}_1}||p_1)) \notag \\
&+ \frac{ \gamma}{2}\min (D(p_{\widetilde{X}_2}||p_0) + D(p_{\widetilde{X}_2}||p_1))\notag \\
&=  (1-\frac{2k}{n})(\sqrt{a} - \sqrt{b})^2 - 2 \gamma\log(\sum_{x\in\mathcal{X}}\sqrt{p_0(x)p_1(x)}) 
\label{eq:theta_star_lower_bound}
\end{align}
Next, we show that the lower bound is achievable.
When the probability distribution of $\widetilde{X}_1, \widetilde{X}_2$ takes the form given by \eqref{eq:p012},
$\epsilon = 0$, and $\theta^*_k$ in \eqref{eq:theta_star} is exactly the lower bound \eqref{eq:theta_star_lower_bound}.
Since $\theta^*_k>0$,
and $|\mathcal{P}_{km}|\leq (km+1)^{|\mathcal{X}|} $, we have $P(A_k) \leq n^{-k(\theta^*_k+o(1))}$.

Using the union bound, we can control $P(F_k)$ by
$$
P(F_k) \leq \binom{n/2}{k}^2 P(A_k)
$$
When $k \geq \frac{n}{\sqrt{\log n}}$, using Lemma 8 of \cite{feng2021},
$P(F_k)$ decreases exponentially. The error probability for $k < \frac{n}{\sqrt{\log n}}$
is analyzed using \eqref{eq:FAk}.
\begin{align*}
&P_e \leq (1+o(1))\sum_{k=1}^{\frac{n}{\sqrt{\log n}}} P(F_k) \leq (1+o(1)) \cdot \\
&\sum_{k=1}^{\frac{n}{\sqrt{\log n}}} \exp(k(-\mu \log n + \frac{2k}{n} \log n(\sqrt{a} - \sqrt{b})^2 - 2\log 2k + 2))
\end{align*}
where $\mu = (\sqrt{a} - \sqrt{b})^2-2 + \gamma D_{1/2}(p_0||p_1) > 0$.
Using the inequality
$$
\frac{2k}{n}(\sqrt{a} - \sqrt{b})^2\log n -2\log2k+2\leq \frac{\mu}{2} \log n
$$
for $1\leq k \leq \frac{n}{\sqrt{\log n}}$, we can obtain
\begin{align*}
P_e &\leq(1+o(1)) \sum_{k=1}^{\frac{n}{\sqrt{\log n}}} \exp(k(-\mu \log n/2)) \\
& =(1+o(1)) \frac{n^{-\mu / 2}}{1-n^{-\mu / 2}} = (1+o(1))n^{-\mu / 2}
\end{align*}
Therefore, \eqref{eq:PeMain} is established.
\newpage
The achievability part can be proved by our SDP algorithm, the details of which will be given in  Section \ref{eq:sdp}.
In the following we show that exact recovery is not possible if $\gamma D_{1/2}(p_0||p_1) + (\sqrt{a} - \sqrt{b})^2 < 2$. Most part of the arguments follow a similar manner to those in \cite{abbe2017community}. Those similar arguments are briefly stated due to space limitation. 

Let $A$ and $B$ be the set of nodes with labels $1$ and $-1$, respectively. Let $H_1$ be a subset of $A$ of size $\frac{n}{\log^3 n}$. It was shown in \cite{abbe2015exact} that in the SBM, with probability at least $\frac{9}{10}$, no node in $H_1$ is connected to more than $\frac{\log n}{\log\log n}$ edges in $H_1$. Similarly, let $H_2$ be a subset of $B$ of size $\frac{n}{\log^3 n}$. Then, with high probability $H_2$ satisfies the same property as $H_1$ does. 

For any $i_1\in H_1$ and $i_2\in H_2$, let $F(i_1,i_2)$ be the event that the likelihood $P(X,Z|(Y_1,\ldots,Y_n))$ (defined in \eqref{eq:lh}) with $Y_{i_1}=-1,Y_{i_2}=1$ is larger than the likelihood $P(X,Z|(Y_1,\ldots,Y_n))$ with $Y_{i_1}=1,Y_{i_2}=-1$, which results in  recovery failure because the maximum likelihood estimate is equivalent to maximum a posteriori probability estimate when $Y$ is uniformly distributed. For a node $i$ and a node set $S\subset\{1,\ldots,n\}, $let $E(i,S)$ be the number of edges between $i$ and nodes in $S$. Then, according to \eqref{eq:lh}, the event $F(i_1,i_2)$ is equivalent to  
the following 
\begin{align}
&\sum_{j=1}^{m} \log \frac{p_1(x^{i_1}_{j})}{p_0(x^{i_1}_{j})}
+\sum_{j=1}^{m} \log \frac{p_0(x^{i_2}_{j})}{p_1(x^{i_2}_{j})}\nonumber\\
\ge &\log \frac{p(1-q)}{q(1-p)}(E(i_1, A \backslash \{i_1\}) + E(i_2, B \backslash \{i_2\})\nonumber\\
&- E(i_1, B \backslash \{i_2\}) - E(i_2, A \backslash \{i_1\})) \label{eq:F1ij}.
\end{align}
By the definition and property of $H_1$ and $H_2$ mentioned above, the probability of \eqref{eq:F1ij} is lower bounded by the probability of
\begin{align}
&\sum_{j=1}^{m} \log \frac{p_1(x^{i_1}_{j})}{p_0(x^{i_1}_{j})}
+\sum_{j=1}^{m} \log \frac{p_0(x^{i_2}_{j})}{p_1(x^{i_2}_{j})}\nonumber\\
\ge &q\log \frac{p(1-q)}{q(1-p)}(2\frac{\log n}{\log\log n}E(i_1, A \backslash \{i_1\}) + E(i_2, B \backslash \{i_2\})\nonumber\\
&- E(i_1, B \backslash \{i_2\}) - E(i_2, A \backslash \{i_1\})) \label{eq:G1ij}.
\end{align}
with high probability. Let $G(i_1,i_2)$ be the event that \eqref{eq:G1ij} holds for any $i_1\in H_1$ and $i_2\in H_2$. In the next, we show that
the probability of $G(i_1,i_)$ is lower bounded by $\frac{1}{n^{-2+\mu}}$ for some positive constant $\mu$. Then we have that 
\begin{align}
& P(\cap_{i_1\in A, i_2\in B} F^c(i_1,i_2)) = (1 - P(F(i_1,i_2)))^{|H_1|\cdot|H_2|} \nonumber\\
& \le (1 - P(G(i_1,i_2)))^{|H_1|\cdot|H_2|}\nonumber\\
& \leq (1-n^{-2+u})^{n^2/\log^6 n} \nonumber\\
& \leq \exp(-n^{\mu+ o(1)}) \to 0, \label{eq:boundg}
\end{align}
where $F^c(i_1,i_2)$ is the event that $F(i_1,i_2)$ does not occur. Eq. \eqref{eq:boundg} implies that with high probability $F(i_1,i_2)$ occurs for some $i_1\in A$ and $i_2\in B$, and hence the failure occurs.

Note that the right hand side of \eqref{eq:G1ij} can be written as $\log \frac{p(1-q)}{q(1-p)}(Z_1-Z_2+o(\log n))$, where $Z_1\sim Binom(n-2,\frac{a\log n}{n})$ and $Z_2\sim Binom(n-2,\frac{b\log n}{n})$ 
follow binomial distributions, where a binomial distribution $Binom(s,r)$ is the probability distribution of the sum of $s$ independent Bernouli random variables, each equals $1$ with probability $r$.
The left hand side can be expressed as
\begin{align*}
\epsilon\triangleq&m\log^{-1}( a /b)\cdot [(D(P_{\widetilde{X}^{i_1}} || P_0) - D(P_{\widetilde{X}^{i_1}} || P_1)) \\
&+(D(P_{\widetilde{X}^{i_2}} || P_1) - D(P_{\widetilde{X}^{i_2}} || P_0))],
\end{align*}
where $P_{\widetilde{X}^i}$ is the empirical distribution of samples $\{x^i_j\}^m_{j=1}$ at node $i$. Let $P_{\widetilde{X}^{i_1}}$ and $P_{\widetilde{X}^{i_2}}$ follow the distribution specified in \eqref{eq:p012}. Then we have that $\epsilon =0$. Using Sanov's theorem and Lemma 4 from \cite{abbe2015exact}, we have that
\begin{align*}
&P(G_1(i,j))\\
\geq &\frac{1}{(m+1)^{2|\mathcal{X}|}} \exp(-m(D(p_{\widetilde{X}_1} || p_0) + D(p_{\widetilde{X}_2} || p_1)) \\
&\cdot\exp(- (\sqrt{a} - \sqrt{b})^2\log n+o(\log n) ) \\
& = \exp(-\log n (\gamma D_{1/2}(P_0||P_1) + (\sqrt{a} - \sqrt{b})^2+ o(1))),
\end{align*}
where $\mathcal{X}$ is the alphabet of distributions $P_0$ and $P_1$.
Since $\gamma D_{1/2}(P_0||P_1) + (\sqrt{a} - \sqrt{b})^2<2$, we have that $P(G_1(i,j))\ge n^{-2+\mu}$ for some positive constant $\mu$.
\end{proof}

\begin{proof}[Proof of Theorem \ref{thm:sdp}]
Consider the dual problem of \eqref{eq:sdp}
\begin{align*}
\min &\sum_{i=1}^{n+1} y_i \\
s.t. &\, \diag\{y_1, \dots, y_{n+1}\} + \Xi - \widetilde{B} \succeq 0, 
\end{align*}
where the $(n+1)\times (n+1)$ symmetric matrix $\Xi$ is defined as 
\begin{equation}
\Xi_{ij} = \begin{cases}
0, & i=j \\
\lambda_i + \lambda_j, & i\neq j
\end{cases},
\end{equation}
Similar to the arguments in \cite{abbe2015exact}, to prove that  

We choose $y_i, \lambda_i$ for $i=1,\dots, n+1$ which satisfy
the dual constraint and achieve the strong duality.
We define $\mu=\frac{1}{n}\mathbbm{1}_n^T h$ and $\lambda = -u/n$.
Then we choose $\lambda_1=\mu, \lambda_{i+1}=g_i\lambda + \lambda$
and $y_i$ are chosen such that
$(\diag\{y_1, \dots, y_n\} + \Xi - \widetilde{B})\tilde{g}=0$, from which we obtain
\begin{align}
y_1 &= h^T g - n\lambda \\
y_{i+1} & = (h_i +\lambda)g_i + \lambda + \frac{1}{2}\diag\{Bgg^T\}, i = 1, \dots, n
\end{align}
We can also verify that $\sum_{i=1}^{n+1} y_i = 2h^Tg +\frac{1}{2} g^T B g$.
The condition for such solution pair to become optimal is then
\begin{align}
\diag(y) + \Xi - \widetilde{B} & = \begin{pmatrix} h^T (g+\frac{1}{n}\mathbbm{1}_n) & -h^T +(\mu + \lambda)\mathbbm{1}_n^T + \lambda g^T\\
-h+(\mu + \lambda )\mathbbm{1}_n + \lambda g& \Xi_n \end{pmatrix}
\succeq 0 \notag \\
\textrm{ where }\Xi_n & = \diag(hg^T + \frac{1}{2}Bgg^T - \lambda \mathbbm{1}_ng^T)  + 2\lambda J_n -\lambda I_n - \frac{1}{2}B + 2\lambda \Xi'
\end{align}
The matrix $\Xi'$ satisfies $\Xi'_{ij}=g_i + g_j$.

By the construction, $\tilde{g}$ is the eigenvector of $\diag(y) + \Xi - \widetilde{B}$ with eigenvalue $0$.
If all eigenvalues of $\Xi_n$ is larger than zero, then by
Cauchy's Interlace Theorem, all eigenvalues of $\diag(y) + \Xi- \tilde{B}$ is larger than zero.
Let $A$ be the adjacency matrix of $G$
and we define $J_n = \mathbbm{1}_n \mathbbm{1}_n^T $, then $B=2A-J_n+I_n$.
Then we have
\begin{equation}\label{eq:Xi_n}
\Xi_n = \diag(hg^T + Agg^T -\lambda \mathbbm{1}_ng^T)  + (\frac{1}{2} +2\lambda) J_n -\lambda I_n - A + 2\lambda \Xi'
\end{equation}
For any vector $x \in \mathbb{R}^n$ with $\norm{x}=1$, we decompose it as $x=\frac{\beta}{\sqrt{n}} g
+ \sqrt{1-\beta^2} g^{\perp}$ where $g^Tg^{\perp}=0, \beta \in [0,1], \norm{g^{\perp}}=1$, we can expand $x^T \Xi_n x$ as
\begin{align*}
x^T \Xi_n x = \frac{\beta^2}{n} g^T \Xi_n g  &
+		\frac{\beta}{\sqrt{n}}\sqrt{1-\beta^2} g^T \Xi_n g^{\perp}
\\
&+
(1-\beta^2)(g^{\perp})^T \Xi_n g^{\perp} 
\end{align*}
For the first term, using $(\diag(Bgg^T) - B)g=0$ we have
\begin{align*}
g^T \Xi_n g = g^T(h -\lambda I_n) -\lambda  = g^T h-\lambda
\end{align*}
Since $\mathbb{E}[g_ih_i]$ is a positive number of order $O(\log n)$ while $\lambda=O(\frac{\log n}{n})$,
by Sanov's theorem
$P(g^T h < 0)$ decreases exponentially. Next we analyze the second term, Let $\tilde{h}_i
=n\lambda+h_i-\lambda-\lambda g_i$, then
$g^T \Xi_n g^{\perp} = \tilde{h}^T g^{\perp} \geq -\norm{\tilde{h}-\frac{1}{n}(\tilde{h}^Tg)g}$.
The norm can be expanded as:
\begin{align*}
\norm{\tilde{h}-\frac{1}{n}(\tilde{h}^Tg)g}^2
=\norm{\tilde{h}}^2 - \frac{1}{n}(\tilde{h}^Tg)^2
\end{align*}
We define $\hat{g}_1 = \frac{1}{2}(g + \mathbbm{1}_n)$ and $\hat{g}_2 = \frac{1}{2}(-g +\mathbbm{1}_n)$.
Using $\tilde{h}^T\mathbbm{1}_n=\mu$ 
\begin{align*}
\frac{\norm{\tilde{h}}^2}{n} - (\frac{1}{n}\tilde{h}^Tg)^2
&=\frac{\norm{\tilde{h}}^2}{n} - 2\frac{(\tilde{h}^T \hat{g}_1)^2}{n^2} - 2\frac{(\tilde{h}^T \hat{g}_2)^2}{n^2} + \frac{(\tilde{h}^T\mathbbm{1}_n)^2}{n^2} \\
&=\frac{ \sum_{i<j,i,j\in S_1} (\tilde{h}_i - \tilde{h}_j)^2 + \sum_{i<j,i,j\in S_2} (\tilde{h}_i - \tilde{h}_j)^2 }{n^2}+ \frac{\mu^2}{n^2}\\
& = I_1 + I_2 + \frac{\mu^2}{n^2}
\end{align*}
where $I_i=\frac{\sum_{i\in S_i} h_i^2}{n} - 2\frac{(h^T \hat{g}_i)^2}{n^2}$.

If $i\in S_i$, $\mathbb{E}[h_i]=m D_i, \Var[h_i]=m \bar{D}_i, \mathbb{E}[h_i^2]=\Var[h_i]+\mathbb{E}^2[h_i]=m \bar{D}_i+m^2 D^2_i, \Var[h_i^2] \leq m^4 \bar{C}_i$ where
$D_i, C_i,i=1,2$ are constants irrelevant with $n$.
We bound $I_i$ by Chebyshev's inequality
\begin{align*}
P(\Big| \frac{\sum_{i\in S_i} h_i^2}{n} - \frac{1}{2}(m \bar{D}_i + m^2D_i^2) \Big| \geq \log n) & \leq \frac{m^4 \bar{C}_i}{2n\log^2 n} \\
P(\Big| \frac{h^T \hat{g}_i}{n} - \frac{m}{2}D_i\Big| \geq \log^{-1} n) & \leq \frac{m \log^{2} n\bar{D}_i}{2n}
\end{align*}
Therefore, with probability $1-n^{1-o(1)}$, we have
$$
I_i \leq \frac{1}{2}m\bar{D}_i + \frac{1}{2}m^2 D_i^2 + \log n - 2(\frac{1}{2}m D_i - \log^{-1} n)^2 = O(\log n)
$$

Therefore, the second term is lower bounded by
$$
\frac{1}{\sqrt{n}} g^T \Xi_n g^{\perp} \geq -\sqrt{\frac{\norm{\tilde{h}}^2}{n} - (\frac{1}{n}\tilde{h}^Tg)^2} = O(\sqrt{\log n})
$$
For the last term $(g^{\perp})^T \Xi_n g^{\perp} >0$, it is equivalent
to consider the subspace orthogonal to $g$:
$$
\min_{x \perp g, x \in \mathbb{R}^n } x^T \Xi_n x \geq 0
$$
Note that $\mathbb{E}[A] = \frac{p-q}{2}gg^T + \frac{p+q}{2}J_n - pI_n$,
using \eqref{eq:Xi_n}, we can simplify $x^T \Xi_n x$ for $x \perp g$ as
\begin{align*}
x^T \Xi_n x &= x^T \diag( -\lambda \mathbbm{1}_ng^T+hg^T + Agg^T) x  \\
&+ \frac{1}{2}(1-p-q+2\lambda)x^TJ_n x
+ p -\lambda -x^T(A-\mathbb{E}[A])x
\end{align*}
By Theorem 5.2 of \cite{lei2015consistency},
with probability $1-n^{-r}$, $\lambda_{\max}(A-\mathbb{E}[A]) \leq c\sqrt{\log n}$ for some positive constant $r$ and $c$.
\begin{align*}
x^T \Xi_n x \geq \min\{(-\lambda + h_i) g_i + g_i (Ag)_i \} - c \sqrt{\log n}
\end{align*}
The error probability is then bounded by
\begin{align*}
P(\Xi_{ii} \leq c\sqrt{\log n}, \forall 1\leq i \leq n)
\end{align*}

Since $\mathbb{E}[\Xi_{ii}]=O(\log n)$, the order of $P(\Xi_{ii} \leq c\sqrt{\log n})$ is unchanged when we modify $c=0$. Therefore, it is sufficient to consider $\Xi_{ii} \leq 0 $, which is
\begin{align}\label{eq:equiv_condition}
\sum_{j=1}^{n/2} (z_j - z'_j) +g_i( h_i -\lambda )\leq 0
\end{align}
Since $\lambda = O(\frac{\log n}{n})=o(h_i)$, we also can omit it.

Depending on the sign of $g_i$, \eqref{eq:equiv_condition} can be divided into two cases. When $g_i=1$, the samples consisted in $h_i$
follow distribution $p_0$ and the probability is bounded by $n^{-\theta^*_1 + o(1)}$ where
\begin{align}
\theta^*_1 &= \min_{\widetilde{X}_1} \gamma D(p_{\widetilde{X}_1}|| p_0)+ \frac{1}{2} g(a,b, 2\epsilon) \label{eq:theta_star2} \\
\epsilon &= \gamma \frac{D(p_{\widetilde{X}_1} || P_1) - D(p_{\widetilde{X}_1} || P_0) }{\log a /b}
\end{align}
Using linear approximation of $g$ in \eqref{eq:g_linear}, we can show that 
\begin{align*}
\theta^*_1 \geq \frac{1}{2}((\sqrt{a}-\sqrt{b})^2+\gamma D_{1/2}(p_0||p_1)) \\
\geq \frac{1}{2}((\sqrt{a}-\sqrt{b})^2+\gamma D_{1/2}(p_0||p_1))
\end{align*}
Similar results are obtained when $g_i=-1$. Finally,
$P(\Xi_{ii} \leq c\sqrt{\log n}, \exists 1\leq i \leq n)$ is bounded by $n^{1-\theta_1^*+o(1)}$, and Theorem \ref{thm:sdp} follows.
\end{proof}

\section{SDP Relaxation}
In this section, we propose a semi-definite programming based relaxation to find the maximum likelihood estimate of the labels $Y$. Note that
 the ML estimate of the labels $Y$ is NP-hard. 
Our SDP algorithm finds the true label $Y^*$ with probability approaching $1$, if
$\gamma D_{1/2}(p_0||p_1) + (\sqrt{a} - \sqrt{b})^2 > 2$.
The SDP problem can be implemented using various iterative schemes

Let $A(Y)$ and $B(Y)$ be the sets of nodes with label $1$ and $-1$, respectively.
According to \eqref{eq:lh}, the log likelihood of $Y$ is given by
\begin{align}\label{eq:loglh}
&\sum^m_{j=1}[\sum_{i\in A(Y)} \log P_1(X^i_{j})+\sum_{i\in B(Y)} \log P_0(X^i_{j})]\nonumber\\
&+\sum_{i,j\in A(Y),i<j}[z_{i,j}\log p+(1-z_{i,j})\log (1-p)]\nonumber\\
&+\sum_{i,j\in B(Y),i<j}[z_{i,j}\log p+(1-z_{i,j})\log (1-p)]
\nonumber\\
&+\sum_{i\in A(Y),j\in B(Y)}[z_{i,j}\log q+(1-z_{i,j})\log (1-q)].
\end{align}
It can be verified that maximizing \eqref{eq:loglh} over $Y\in\{\pm\}^n$ is equivalent to the following optimization problem
\begin{align}
\max_{x}\, & h^T v + \frac{1}{2}v^T B v \notag \\
s.t.\, & \mathbbm{1}_n^T v = 0 \text{ and } v_i \in \{\pm 1 \} \label{eq:matrix_mle}
\end{align}
where $h$ is an n-dimensional vector with entry $h_i = \frac{1}{\log \frac{p(1-q)}{q(1-p)}}\sum_{j=1}^m \log \frac{P_0(x^i_{j})}{P_1(x^i_{j})}$ for $i\in\{1,\ldots,n\}$ and the $n\times n $ matrix $B$ is defined as
\begin{equation}
B_{ij} = \begin{cases}
1, & (i,j)\in E(G) \\
 -1, (i,j) \not\in E(G)
\end{cases}
\end{equation}  
for $i,j\in\{1,\ldots,n\}$.
%However, we find that
%the maximal solution is unchanged when $\kappa$ takes other values as long as $\kappa > b\frac{\log %n}{n}$.

Let $v^*$ be the optimal solution to \eqref{eq:matrix_mle} and $V=(1,v^*)(1,v^*)^T$, where $(1,v^*)$ is a $(n+1)$-dimensional  vector obtained by concatenating $1$ and $v^*$.
Then, the optimal value of \eqref{eq:matrix_mle} equals $\frac{1}{2}Tr(\widetilde{B}V)$, where
\begin{equation}\label{eq:B_lambda_def}
\widetilde{B} = \begin{pmatrix} 0 & h^T  \\ h  & B \end{pmatrix}.
\end{equation}
We wish to show that $V$ is the unique optimal solution to the following problem.
\begin{align}
\max\, Tr(\tilde{B}V)  &\notag\\
s.t.\, V_{ii} &= 1, \notag\\
 V &\succeq 0, \notag\\
 \sum_{j=1,j\neq i}^{n+1} (V_{ij} + V_{ji})& = 0,~\forall i\in\{1,\ldots,n\}. \label{eq:sdp}
\end{align}
Note that here we use $n+1$ constrains in the last line of \eqref{eq:sdp} to describe the balanced property of the labels. This is important in deriving the optimality and uniqueness of $V$, which will be proved in the following theorem. The theorem shows that our SDP relaxation achieves exact recovery with high probability, if the threshold is met.
%When $m=0$ and $\kappa=1$,
%the SDP is exactly the one Abbe considered in \cite{abbe2015exact}.
%When $m=0$ and $\kappa=\frac{a-b}{\log a/b}\frac{\log n}{n}$, the SDP is
%the same as the one in Theorem 3 of \cite{Hajek16}.
% The theoretical guarantee for the relaxation form in \eqref{eq:sdp} is summarized
% in the following theorem:
\begin{theorem}\label{thm:sdp}
	If $\gamma D_{1/2}(p_0||p_1)  + (\sqrt{a} - \sqrt{b})^2 > 2$, then with high probability, the optimal solution
	 $V^*$ to \eqref{eq:sdp} is unique and given by $(1,y^*)(1,y^*)^T$, where $y^*$ is the true labeling of the nodes.
 \end{theorem}
% , and Theorem \ref{thm:sdp} gives a quantification of its theoretical
% performance. For observations from binary symmetric channel, the result is
% \begin{corollary}
% 	Suppose $X_{ij}$ is the output of $Y_i$ from a binary symmetric channel with crossover probability
% 	$p^*$. $P(\hat{Y}_{\SDP} \neq  Y) = o(1)$ as long as
% 	\begin{equation}
% 	-2\log(2\sqrt{p^*(1-p^*)}) + (\sqrt{a} - \sqrt{b})^2>2
% 	\end{equation}
% \end{corollary}
\section{Conclusion}\label{s:conclusion}
In this paper, we obtain a close-form exact recovery condition for a two-community SBM with side information. This condition
shows that the detection error can be characterized by Rényi divergence and the parameters of SBM. To control the recovery error within a given level,
our result shares insight on sample complexities of node features.


\bibliographystyle{IEEEtran}
\bibliography{exportlist}
\end{document}